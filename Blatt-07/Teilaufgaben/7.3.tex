\begin{enumerate}
	\item Für den Kandidaten ist es sinnvoll das Tor nicht zu wechseln. Folgende Verteilungen sind nämlich möglich:
	\begin{itemize}
		\item Hinter 1 ist das Auto, Tür 3 öffnet weil $ 2<3$ 
		\item Hinter 1 ist das Auto, Tür 3 offnet weil es kein Gewinn ist
		\item Hinter 2 ist das Auto, Tür 3 öffnet weil nicht gewählt 
		\item in 2 von 3 Fällen gewinnt der Kandidat also 
		\item Das Auto liegt mit \(P= \frac{2}{3} \) hinter Tür 1
		\item Bei einer Umentscheidung sinken die Chancen auf \( \frac{1}{3}\) ab.
	\end{itemize}

	\item für den Kandidaten B ist es sinnvoll zu wechseln:
	\begin{itemize}
		\item Es konnten Tor 2 und 3 geöffnet werden
		\item Der Showmaster öffnet im zweifelsfall das größere Tor
		\item Tor 2 wurde geöffnet
		\item Wäre hinter 3 auch eine Ziege müsste er Tor 3 öffnen
		\item Das Auto ist hinter Tor 3 ($P=1$)
		\item Eine Umentscheidung führt zum sicheren Gewinn
	\end{itemize}
\end{enumerate}
\textbf{ALTERNATIVE:}\\
Im Folgenden ist das Ereignis $A$ \glqq die Umentscheidung führt zum Erfolg'' und das Ereignis $B$ \glqq die erste Wahl war richtig''.
\begin{enumerate}
	\item[a)] Da vom Moderator das Tor mit der größten Nummer gewählt wird um den \glqq Zonk'' zu enthüllen - hier ist das Tor 3 - kann sich der Hauptpreis weiterhin hinter Tor 1 oder Tor 2 verbergen. Damit ergeben sich folgende Wahrscheinlichkeiten
		\begin{itemize}
			\item Der Erfolg der Umentscheidung unter der Bedingung, dass das gewählte Tor bereits richtig war: $P(A|B) = 0$
			\item Die Wahrscheinlichkeit, dass der Preis hinter der Erstwahl ist: $P(B)=\frac{1}{3}$
			\item Die Erfolgswahrscheinlichkeit unter der Bedingung, dass der Preis nicht hinter der ersten Wahl verborgen ist und der Spieler sich umentscheidet: $P(A|B^c)=1$
			\item Die Wahrscheinlichkeit dafür, dass der Preis nicht hinter der Erstwahl verborgen ist: $P(B^c)=\frac{2}{3}$
			\begin{equation*}
				\begin{split}
					P(A) 	&= P(A|B)P(B) + P(A|B^c)P(B^c)\\
							&= 0 \cdot \frac{1}{3} + 1 \cdot \frac{2}{3}\\
							&= \frac{2}{3}
				\end{split}
			\end{equation*} 
		\end{itemize}
	\item[b)] Der Moderator enthüllt nun den \glqq Zonk'' hinter dem Tor 2. Da auch weiterhin das Tor mit der größten Nummer gewählt wird, das nicht den Hauptpreis enthält und der Spieler Tor 1 mit seiner Wahl blockiert, muss sich der Hauptpreis hinter Tor 3 befinden. Damit ergeben sich folgende Wahrscheinlichkeiten
		\begin{itemize}
			\item Der Erfolg der Umentscheidung unter der Bedingung, dass das gewählte Tor bereits richtig war: $P(A|B) = 0$
			\item Die Wahrscheinlichkeit, dass der Preis hinter Tor 1 verborgen ist, mit dem o.g. Wissen: $P(B) = 0$
			\item Die Erfolgswahrscheinlichkeit unter der Bedingung, dass der Preis nicht hinter der ersten Wahl verborgen ist und der Spieler sich umentscheidet: $P(A|B^c)=1$
			\item Aufgrund des Auswahlverfahrens des Moderators beträgt die Wahrscheinlichkeit, dass sich der Preis nicht hinter der Erstwahl verbirgt und bereits ein Tor enthüllt wurde: $P(B^c) = 1$
			\begin{equation*}
				\begin{split}
					P(A) 	&= P(A|B)P(B) + P(A|B^c)P(B^c)\\
							&= 0 \cdot 0 + 1 \cdot 1\\
							&= 1
				\end{split}
			\end{equation*}

		\end{itemize}
\end{enumerate}
