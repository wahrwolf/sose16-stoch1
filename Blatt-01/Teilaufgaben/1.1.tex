a) Die Bestimmung der Sonnenaktivität ist kein Zufallsexperiment, da sie leicht vohergesagt werden kann. Die Aktivität ist zyklisch, leicht zu beobachten und sogar für einen Laien (nicht Meterolge) gut zu beschreiben

b) Die Bestimmung der Verkehrsituation Donnerstags ist ebenfalls kein Zufallsexperiment. Auch wenn sie nicht ganz so zyklisch und regelmäßig wie die Sonnenaktivität ist, so lässt sich die Situation an einem Ort relativ präzise beschreiben.
Da es sich auch um eine Betrachtung mehrerer Tage handelt lässt sich die Tendenz gut vohersagen.

c) Hierbei handelt es sich um ein Zufallsexperiment, da die beeinflussenden Faktoren zu komplex und/oder zu schwer zu beobachten sind. Da allerdings die Augensumme beobachtet wird lässt sich tatsächlich auch gegen ein Zufallsexperiment argumentieren. Je mehr Würfel am Wurf beteiligt werden, desto einfacher wird es das Ergebnis vorherzussagen (oder es zumindest einzugrenzen)

d) In meinen Augen handelt es sich auch hierbei um ein Zufallsexperiment. Hierfür sehe ich im Großen 3 Punkte die dafür sprehcen. Zum einen gibt es ein gewisses Spektrum an Möglichkeiten von unter 1 Jahr bis über 10 Jahre. Außerdem ist die Verteilung relativ gleichmäßig. Zum Schluss zeigt sich aber auch dass die Anzahl der Faktoren zu groß oder die Verstrickung in politische Machenschaften zu komplex ist/war um die restliche Lebenszeit abzuschätzen.

