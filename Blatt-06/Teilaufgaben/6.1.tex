\begin{enumerate}
	\item[a)] Der erwartete Gewinn im ''einfachen'' Roulettespiel wird wie folgt berechnet:
	%Der Erwartungswert für den 'einfachen' Roulettefall setzt sich aus folgenden Bestandteilen zusammen:
	\begin{itemize}
		\item Für Rot und Schwarz gilt jeweils eine Wahrscheinlichkeit von $\frac{18}{37} $
		\item Für die grüne Null gilt eine Wahrscheinlichkeit von $\frac{1}{37}$
		\item Da nur auf ''rot'' oder ''schwarz'' gesetzt wird, beträgt die Auszahlung im Erfolgsfall: $n \cdot \frac{18}{37}$
		\item Im Falle eines Misserfolges beträgt der Verlust: $ n \cdot (\frac{18}{37}+\frac{1}{37})$
		\item Daraus ergibt sich ein erwarteter Gewinn von: $ n \cdot \frac{18}{37} - n \cdot (\frac{18}{37}+\frac{1}{37}) = -\frac{1}{37}n$
		
		%\item daher berechnet sich die erwartete Auszahlung mit: $2n \cdot \frac{18}{37}=n\cdot \frac{36}{37} $
		%\item bei einem Einsatz von $n$ ergibt das einen Erwarungswert von: $n\cdot \frac{36}{37} - n = -\frac{1}{37}$, also einen Verlust von etwa 0.03 Geldeinheiten pro Spiel
	\end{itemize}
	\item[b)] Der erwartete Gewinn im ''allgemeinen'' Roulettespiel berechnet sich wie folgt:
	\begin{itemize}
		\item Für die Gruppe von Zahlen (aus ''rot'' und ''schwarz''), die gesetzt werden, gilt die Wahrscheinlichkeit $\frac{k}{37}$
		\item Für die Gruppe von Zahlen (aus ''rot'' und ''schwarz''), die nicht gesetzt werden, gilt entsprechend $\frac{36-k}{37}$
		\item Die Wahrscheinlichkeit für die ''grüne Null'', die nicht gesetzt werden kann, gilt weiterhin $\frac{1}{37}$
		\item Die Auszahlung im Erfolgsfall beträgt $n\cdot (\frac{36}{k}-1)\cdot \frac{k}{37}$
		\item Der Verlust im Falle des Misserfolges beträgt $n\cdot (\frac{36-k}{37}+\frac{1}{37})$
		\item Der erwartete Gewinn beträgt somit:
		\begin{equation*}
			\begin{split}
				E(G)	&= n\cdot (\frac{36}{k}-1)\cdot \frac{k}{37} - n\cdot (\frac{36-k}{37}+\frac{1}{37})\\
						&= \frac{36}{37}n - \frac{k}{37}n - \frac{36}{37}n + \frac{k}{37}n - \frac{1}{37}n\\
						&= \frac{1}{37}n
			\end{split}
		\end{equation*}
	\end{itemize}
	
	%Der Erwartungswert mit dem Setzten von $k$ Zahlen berechnet sich mit: $n\cdot\frac{36}{k} \cdot \frac{k}{37} -n=-\frac{1}{37}$
	\item[c)] Der Erwartungswert des Gewinns hängt somit nur von $n$ ab, mit der Wahrscheinlichkeit von $1/37$, dass die ''grüne Null'' getroffen wird. Da die ''grüne Null'' nicht gesetzt werden kann, ist der Gewinn negativ. Der Erwartungswert ist unabhängig von $k$.

\end{enumerate}
