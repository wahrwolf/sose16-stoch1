\begin{enumerate}
	\item[a)] Im Falle einer Auszahlung von $k$ beträgt der Erwartungswert:
		\begin{equation*}
			\begin{split}
				E(X) 	&= \sum_{k=1}^{n} k \cdot P(X = k)\\
						&= \sum_{k=1}^{n} k \cdot \frac{1}{k(k+1)}\\
						&= \sum_{k=1}^{n}\frac{1}{k+1}
			\end{split}
		\end{equation*}
	\item[b)] Im Falle einer Auszahlung von $k^2$ beträgt der Erwartungswert:
				\begin{equation*}
					\begin{split}
						E(X) 	&= \sum_{k=1}^{n} k^2 \cdot P(X = k)\\
								&= \sum_{k=1}^{n} k^2 \cdot \frac{1}{k(k+1)}\\
								&= \sum_{k=1}^{n}\frac{k}{k+1}
					\end{split}
				\end{equation*}
	\item[c)] Im Falle einer Auszahlung von $\frac{8}{k+2}$ beträgt der Erwartungswert:
				\begin{equation*}
					\begin{split}
						E(X) 	&= \sum_{k=1}^{n} \frac{8}{k+2} \cdot P(X = k)\\
								&= \sum_{k=1}^{n} \frac{8}{k+2} \cdot \frac{1}{k(k+1)}\\
								&= 8 \sum_{k=1}^{n}\frac{1}{k(k+1)(k+2)}\\
								&= 8 \frac{n(n+3)}{4(n+1)(n+2)}\\
								&= \frac{2n+6}{(n+1)(n+2)}
					\end{split}
				\end{equation*}
\end{enumerate}