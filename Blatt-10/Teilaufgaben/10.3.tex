Die Wahrscheinlichkeit das eine Münze auf einem schwarzen Feld landet setzt sich wie folgt zusammen:

\[ P_{schwarzes~Feld} \cdot P_{liegt~ganz~auf~einem Feld} \]

Ein normales Schachbrett besteht aus 32 schwarzen, sowie 32 weißen Feldern. Daher ist \(P_{schwarzes~Feld}=\frac{32}{64}=\frac{1}{2}\) 

Wenn die Münze einen Durchmesser von 1cm (und somit einen Radius von 0.5cm) hat und ein Schachfeld ein Quadrat mit Seitenlänge 4cm ist, so muss der Mittelpunkt der Münze stets mindestens 0.5cm vom Rand entfernt sein.

Somit bleibt ein Quadrat mit der Seitenlänge von 3cm als mögliche Liegefläche für den Mittelpunkt der Münze. Also gilt für \( P_{liegt~ganz~auf~einem~Feld} = \frac{3^2}{4^2} = \frac{9}{16} \)

Die Gesamtwahrscheinlichkeit liegt also bei \( 0.5 \cdot \frac{9}{16} = \frac{9}{32} \approx 28.1\% \)


