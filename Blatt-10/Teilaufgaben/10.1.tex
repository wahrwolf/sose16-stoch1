\begin{enumerate}
	\item[a)] Fall 1:	$P(Y_{n} = 0)$\\
		\begin{equation*}
			\begin{split}
				|Y_n - E[Y_n]| &\ge \epsilon\\
				|- \frac{n}{2}| &\ge \epsilon\\
				\frac{n}{2} &\ge \epsilon\\
			\end{split}
		\end{equation*}
			Fall 2:	$P(Y_n = n)$
		\begin{equation*}
			\begin{split}
				|Y_n - E[Y_n]| &\ge \epsilon\\
				|n - \frac{n}{2}| \ge \epsilon\\
				\frac{n}{2} \ge \epsilon\\
			\end{split}
		\end{equation*} 
		Damit muss $\epsilon \leq \frac{n}{2}$ gewählt werden.
	\item[b)]
		\begin{equation*}
			\begin{split}
				P(|Y_n - E[Y_n]|\ge \epsilon) 		&\leq \frac{Var[Y_n]}{\epsilon^2}\\
				P(|Y_n - \frac{n}{2}|\ge \epsilon) 	&\leq \frac{\frac{n}{4}}{(\frac{n}{2})^2}\\
													&\leq \frac{1}{n}\\
			\end{split}
		\end{equation*}
		Gemäß der Aufgabenstellung sollen die Werte aus b) verwendet werden. Wir gingen an dieser Stelle davon aus, dass die Präsenzaufgabe 10.2 b) gemeint war. Entsprechend gilt nun $n = 4$
		\begin{equation*}
			\begin{split}
				P(|Y_4 - 2|\ge 2) \leq \frac{1}{4}
			\end{split}
		\end{equation*}
	\item[c)]
		\begin{equation*}
			\begin{split}
				P(Y_n = 0) + P(Y_n = n)		&= B_{n, 0.5}(0) + B_{n, 0.5}(n)\\
											&= {n \choose 0} (0.5)^0 (1-0.5)^{n-0} + {n \choose n} (0.5)^n (1-0.5)^{n-n}\\
											&= 1 \cdot 1 \cdot 0.5^n + 1 \cdot 0.5^n \cdot 1\\
											&= 2 \cdot 0.5^n\\
											&= \frac{1}{2^{n-1}}= 2^{1-n}
			\end{split}
		\end{equation*}
	\item[d)] Sei $f(x) = 2^{n - 1}$ der Nenner des exakten Ergebnisses und $g(x) = n$ der Nenner der Abschätzung.
		\begin{equation*}
			\lim\limits_{n\rightarrow \infty} |\frac{2^{n-1}}{n}| \stackrel{(*)}{=} \lim\limits_{n \rightarrow \infty} |\frac{n2^{n-2}}{1}| = \infty
		\end{equation*}
		Damit wächst der Nenner des exakten Ergebnisses schneller als der aus der Abschätzung. Die Abschätzung liefert damit insbesondere für große $n$ keine gute Annäherung.		
\end{enumerate}


%\begin{enumerate}
%	\item[a)] 
%		\begin{equation*}
%			\begin{split}
%				P(|Y_{n} - E[Y_{n}]| \ge \epsilon)	&= P(Y_{n} = 0) + P(Y_{n} = n)\\
%										&= B_{n,0.5} (0) + B_{n,0.5} (4)\\
%										&= {n\choose 0} (0.5)^{0} (1- 0.5)^{n - 0} + {n \choose n} (0.5)^{n} (1-0.5)^{n - n}\\
%										&= 1 \cdot 1 \cdot 0.5^n + 1 \cdot 0.5^n \cdot 1\\
%										&= 2 \cdot 0.5^n\\
%			\end{split}
%		\end{equation*} 
%		\begin{equation*}
%			\begin{split}
%				\frac{Var[Y]}{\epsilon^2} 		&\ge 2 \cdot 0.5^n\\
%				\frac{n}{4 \cdot \epsilon^2}	&\ge 2 \cdot 0.5^n\\
%				\frac{n}{8 \cdot 0.5^n}			&\ge \epsilon^2\\
%			\end{split}
%		\end{equation*}
%		Fall 1:
%		\begin{equation*}
%			\begin{split}
%				\epsilon \ge \sqrt{\frac{n}{8 \cdot 0.5^n}}
%			\end{split}
%		\end{equation*}
%		Fall 2:
%		\begin{equation*}
%			\begin{split}
%				\epsilon \leq -\sqrt{\frac{n}{8 \cdot 0.5^n}}
%			\end{split}
%		\end{equation*}
%		Fall 2 kann nicht eintreten, da gilt $\epsilon > 0$.
%	\item[b)]
%		\begin{equation*}
%			\begin{split}
%				\epsilon 	&\ge \sqrt{\frac{4}{8 \cdot 0.5^4}}\\
%				\epsilon	&\ge \sqrt{8} \approx 2.8284\\
%			\end{split}
%		\end{equation*}
%		Die Abschätzung mit der Tschebycheff-Ungleichung ergibt:
%		\begin{equation*}
%			\begin{split}
%				P(|Y_{4}- 2 |\ge \sqrt{8}) &\leq \frac{1}{\sqrt{8}^2} = \frac{1}{8} 
%			\end{split}
%		\end{equation*}
%	\item[c)] Weiterhin angenommen, die Werte aus b) gelten (insb. $n=4$), dann erhalten wir einen exakten Wert von:
%		\begin{equation*}
%			\begin{split}
%				P(Y_4 = 0) + P(Y_4 = 4)	&= 2 \cdot 0.5^4 = \frac{1}{8}
%			\end{split}
%		\end{equation*}
%	\item[d)]
%\end{enumerate}