\begin{enumerate}
	\item Bestimmen Sie das Urbild für alle $ x \in \Omega $  \\
		\begin{tabular}{|c|c|}
		\hline
		$X^{-1}(\{x\}) $ & $x$ \\ \hline
		1 & $\{(i,1), (1,j) | i,j \in \{1,2,3,4,5,6\} \}$ \\ \hline
		2 & $\{(i, 2) (2,j) | i,j \in \{2,3,4,5,6 \} \}$ \\ \hline 
		3 & $\{(i, 3) (3,j) | i,j \in \{3,4,5,6 \} \}$ \\ \hline
 		4 & $\{(i, 4) (4,j) | i,j \in \{4,5,6 \} \}$ \\ \hline 	
		5 & $\{(i, 5) (5,j) | i,j \in \{5,6 \} \}$ \\ \hline 	
		6 & $\{(6,6) \}$ \\ \hline 
		\end{tabular}
	\item 	Die Wahrscheinlichkeiten für alle $x$ berechnet sich aus: $ \frac{|X^{-1}(\{x\})|}{n} $ mit $n = 2^{\Omega}$ mit $n = 2^{\Omega}$ \\
		\begin{tabular}{|c|c|c|c|c|c|c|}
		\hline
		$x$  & 1& 2& 3& 4& 5& 6 \\ \hline
		$P(X=x)$ & $ \frac{11}{36}$ & $\frac{9}{36}$& $\frac{7}{36}$& $\frac{5}{36}$& $\frac{3}{36}$& $\frac{1}{36}$ \\ \hline
		\end{tabular}
	\item 	Da alle Einträge in der oberen Tabelle disjunkt sind, gilt:  $P(X \leq 3)=\frac{11+9+7}{36}=\frac{27}{36}=\frac{3}{4}$ \\
		Um die Wahrscheinlichkeit einer ungeraden zu berechnen gilt: $P({1,3,5})= \frac{11 + 7 +3}{36}  =  \frac{21}{36}=\frac{7}{12}$ \\

\end{enumerate}
