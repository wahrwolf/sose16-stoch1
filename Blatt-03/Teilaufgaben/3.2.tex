\textbf{Hausübung 3.2} (Archäologie, 10 Punkte).\\
Laut Aufgabenstellung ist folgende Tabelle mit Wahrscheinlichkeitsmaßen gegeben, mit $R=Reis, P=Peking-Ente$ und $B=Brokkoli$:\\

\begin{tabular}{|c|c|c|c|c|c|c|c|}
		\hline $A$ & $\{R\}$ & $\{P\}$ & $\{B\}$ & $\{R, P\}$ & $\{R, B\}$ & $\{P, B\}$ & $\{R, P, B\}$ \\
		\hline $P(A)$ & $\frac{4}{5}$ & $\frac{1}{2}$ & $\frac{1}{4}$ & $\frac{9}{20}$ & $\frac{3}{20}$ & $\frac{1}{20}$ & $\frac{1}{20}$\\
		\hline
\end{tabular}

Die Wahrscheinlichkeit, dass wenigstens eine der drei Zutaten enthalten ist, berechnet sie wie folgt:\\
\begin{equation*}
	\begin{split}
	P(\{R\}\cup \{P\}  \cup \{B\}) = &P({R}) + P(\{P\}) + P(\{B\}) - P(\{R\}\cap \{P\}) - P(\{R\}\cap \{B\}) - \\
	&P(\{P\}\cap \{B\}) + P(\{R\}\cap \{P\} \cap \{B\})\\
	= &\frac{4}{5} + \frac{1}{2} +\frac{1}{4} - \frac{9}{20} - \frac{3}{20} - \frac{1}{20} + \frac{1}{20}\\
	= &\frac{19}{20}
	\end{split}
\end{equation*}
