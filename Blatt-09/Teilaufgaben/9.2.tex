\begin{enumerate}
	\item[a)] Bei dem angegebenen Graphen handelt sich um eine Darstellung mit einem positiven Korrelationskoeffizenten. Zieht man eine Gerade durch die Punkte, zeigt dies, dass je weiter außen die Punkte auf der x-Achse liegen - also in größeren Wertebereichen von beispielsweise $X$ - so liegen die die einzelnen Punkte auch auf der y-Achse in höheren Wertebereichen. Dies geht einher mit einem positiven Vorzeichen bezüglich der Korrelation, da ein positiver Anstieg des einen Wertes auch einen positiven Anstieg des anderen Wertes zur Folge hat und umgekehrt.
	\item[b)] Dieser Graph zeigt eine negative Korrelation. Ähnlich der Begründung unter a) ist hier jedoch der Unterschied, dass ein höherer Wert auf der einen Achse zu einem niedrigen auf der anderen Achse führt und umgekehrt. Dies lässt daher auf ein negatives Vorzeichen bei der Korrelation schließen.
\end{enumerate}