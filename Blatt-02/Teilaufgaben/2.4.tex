\begin{tabular}{|l|c|c|c|c|}
	\hline
	Ereignis: & A & B & C & D \\
	\hline
	Beschreibung: & Pasch & Summe < 4 & 7 & 11 \\
	\hline
	Fälle: & 6 & 9 & 6 & 2 \\
	\hline
\end{tabular} 

Für 1): Außerdem ist gegeben: $ P(A \cap B ) = \frac{3}{36} = \frac{1}{12} $

\begin{enumerate}
	\item $P(A \cup B) = P(A) + P(B) - P(A \cap B) = \frac{6+9-3}{36}=\frac{1}{3}$
	\item Wenn A, C und D disjunkt sind, lassen sich die Wahrscheinlichkeiten nach dem Additionssatz berechnen
	\begin{itemize}
		\item Alle Würfe die in A liegen haben eine gerade Augenzahl, da 2n stets gerade ist ...
		\item C und D müssen disjunkt sein, da die Anforderungen disjunkte Anforderungen hat (weil 7=11)
		\item Da alle drei disjunkt sind gilt $P(A \cup C \cup D) = \frac{6+6+2}{36} = \frac{1}{2}$
	\end{itemize}
\end{enumerate}
