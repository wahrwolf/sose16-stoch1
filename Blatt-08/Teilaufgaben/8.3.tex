Gemäß der \textbf{Definition 5.2.1} des Skriptes sind zwei Ereignisse $A, B \subset \Omega$ stochastisch unabhängig, wenn gilt
\begin{equation*}
	P(A \cap B) = P(A) \cdot P(B)
\end{equation*}
\begin{enumerate}
	\item[a)] Es gilt die Wahrscheinlichkeit $P(A \cap B) = \frac{1}{6}$, dass die Augenzahl gerade und durch drei teilbar ist. Zudem sind die Wahrscheinlichkeiten für die Ereignisse \glqq gerade Augenzahl'' $P(A)  = \frac{1}{2}$ und \glqq durch drei teilbar'' $P(B) = \frac{1}{3}$. Damit gilt:
	\begin{equation}
		\begin{split}
			\frac{1}{6} &= \frac{1}{3} \cdot \frac{1}{2}\\
						&= \frac{1}{6}.
		\end{split}
	\end{equation}
	Beide Ereignisse sind somit stochastisch unabhängig.
	\item[b)] Es gilt die Wahrscheinlichkeit $P(A \cap B) = \frac{1}{36}$, dass die Augensumme 6 beträgt und mindestens ein Würfel die 3 als Augenzahl zeigt. Zudem sind die Wahrscheinlichkeiten für die Ereignisse \glqq die Augensumme beträgt 6'' $P(A) = \frac{5}{36}$ und \glqq mindestens ein Würfel zeigt die 3'' $P(B) = \frac{11}{36}$. Damit gilt:
	\begin{equation*}
		\begin{split}
			\frac{1}{36} &\neq \frac{5}{36} \cdot \frac{11}{36}\\
						 &\neq \frac{55}{1296}
		\end{split}
	\end{equation*}
	Damit sind beide Ereignisse stochastisch abhängig voneinander.
\end{enumerate}